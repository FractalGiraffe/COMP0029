\documentclass[12pt,leqno]{report}

\usepackage{ur_thesis}
\usepackage{times}
\usepackage{hyperref}
\usepackage{xcolor}		% make links dark blue
\usepackage{setspace}
\definecolor{darkblue}{rgb}{0, 0, 0.5}
\hypersetup{colorlinks=true,citecolor=darkblue, linkcolor=darkblue, urlcolor=darkblue}
\setlength{\parskip}{0pt}
\onehalfspacing
\begin{document}

\pagenumbering{gobble}

\sloppy
\title{\makebox[\linewidth]{\parbox{\dimexpr\textwidth+2cm\relax}{\centering On the Federated Detection of Hate Speech}}}
\author{\LARGE{Valentin Gorbunov}}
\maketitle

\begin{abstract}

  A brief summary.
  \thefontsize{}
  
\end{abstract}

\pagenumbering{arabic}

\tableofcontents

\chapter{Preface}

\chapter{Introduction}

\section{Problem Outline}

\section{Aims and Goals}
\subsection{Aims}
\subsection{Goals}
\section{Project Overview}
\section{Report Structure}

\chapter{Research Paper}

\chapter{Conclusions and Evaluation}
\section{Achievements}
\section{Critical Review}
\section{Future Work}

\appendix
\chapter{Results}
\chapter{Project Plan}

\begin{description}
\item[Project title:] On the Federated Detection of Hate Speech

\item[Supervisor’s name:] Emiliano De Cristofaro

\item[Aim:] The aim of the project is to design a federated learning system to detect hate speech and investigate it in terms of its utility.

\item[Objectives:] \hfill
\begin{itemize}
\item Review current machine learning approaches to detecting hate speech and survey federated learning software
\item Design a federated learning system to detect hate speech
\item Investigate the proposed system in terms of its utility
\item Integrate findings into existing infrastructure
\end{itemize}
\item[Deliverables:] \hfill
\begin{itemize}
\item Literature review surveying current machine learning approaches to detecting hate speech and federated learning software
\item Architectural specification of federated learning system
\item Implementation of federated learning system and code basis for reproducing experimental results
\item Evaluation of the federated learning system’s utility
\item Strategy for testing and evaluating the system
\item Documented module to integrate with existing infrastructure
\end{itemize}
\item[Work Plan:] \hfill
\begin{itemize}
\item Project start to early-November (6 weeks) Review literature, familiarize with relevant tools and formulate problem statement.

\item Mid-November to early-December (2 weeks) Implement system and setup experiment environment.

\item Mid-December to Mid-February (10 weeks) Conduct experiments and evaluate system.

\item Mid-February to end-March (6 weeks) Work on Final Report.
\end{itemize}
\end{description}

\chapter{Interim Report}
\begin{description}
\item[Project title:] On the Federated Detection of Hate Speech

\item[Supervisor’s name:] Emiliano De Cristofaro

\item[Progress Summary:] \hfill
\begin{description}

\item[Completed:] \hfill
\begin{itemize}
\item Literature review
\item Architectural specification of federated learning system
\item Implementation of system and code basis for reproducing experimental results 
\end{itemize}
\item[In Progress:] \hfill
\begin{itemize}
\item Evaluation of the federated learning system’s utility
\item Strategy for testing and evaluating the system
\end{itemize}
\end{description}

\item[Work Plan:] \hfill
\begin{itemize}
\item Mid-January to Mid-February (4 weeks) Evaluate federated learning system’s utility and improve strategy for testing and evaluating the system.

\item Mid-February to end-March (6 weeks) Work on Final Report.
\end{itemize}
\item[Progress Report:] \hfill

I completed a literature review on automatic hate speech detection, software tools for federated learning and the methods and challenges of federated learning.
The following lines of inquiry developed:
\begin{itemize}
\item How does the key challenge in federated learning, statistical heterogeneity, affect performance?
\item How can methods of federated learning, such as FedBN and FedOpt, address the challenges posed by statistical heterogeneity?
\end{itemize}

I set up an experiment environment, leveraging Google’s ecosystem that includes Jupyter Notebook, Google Colab and Drive to streamline my workflow.

I ported the centralized Davidson pipeline to Python 3 and migrated libraries. I developed a deep understanding of their results and methodology. I broke the pipeline down into components, documented them and adapted the pipeline so it could be federated. I evaluated the centralized pipeline’s utility.

I designed, implemented, and documented a single machine simulation of the federated pipeline using flower and multiprocessing. Baseline experiment shows comparable performance to the centralized pipeline.

I plan to continue working on evaluating the effect of statistical heterogeneity on the performance of the federated learning system. Additionally, I plan to investigate how methods of federated learning can address the challenges posed by statistical heterogeneity.
\end{description}
\end{document}



